\section*{From Racial Preferences to Residential Segregation}
In this section, we introduce the problem of finding an residential
pattern that is most compatible with the racial preferences of
individual households.

Let's imagine a very small city consisting of blocks. Only one house
lives on each block. In this city are two white household and two
black households. The white households prefer to live next to whites
and the black household prefer to live next to blacks. What
residential patterns are most compatible with those preferences?

We can represent this tiny town as a network where each block is
connected to the blocks that front the same street. In this
representation, blocks that are kitty corner are not directly
connected. We'll index the blocks as $1$, $2$, $3$, and $4$.

\begin{figure}[h]
  \centering
  \begin{tabular}{cc}
\tikz{
\draw[help lines] (0,0) grid (2,2);
\node at (0.5, 0.5) {3} ;
\node at (1.5, 1.5) {2} ;
\node at (0.5, 1.5) {1} ;
\node at (1.5, 0.5) {4} ;
}
\\
\tikz{ %
  \node[latent] (1) {$1$} ; %
  \node[latent, below left=of 1] (2) {$2$} ; %
  \node[latent, below right=of 1] (3) {$3$} ; %
  \node[latent, below left=of 3] (4) {$4$} ; %
  \edge[-] {2,3} {1} ; %
  \edge[-] {2,3} {4} ; %
}
\end{tabular}
\end{figure}

In our city, each block is inhabited by only a white household or
only a black household. We will denote the race of the $i$th
block as $y_i$. Between every pair of neighbors, there is a racial
frustration cost $\phi_{i,j}$ indicating how unhappy they are if their
racial preferences are frustrated.

\begin{figure}[!h]
\centering

\tikz{ %
  \node[latent] (1) {$y_1$} ; %
  \node[latent, below left=of 1] (2) {$y_2$} ; %
  \node[latent, below right=of 1] (3) {$y_3$} ; %
  \node[latent, below left=of 3] (4) {$y_4$} ; %
  \factor[below left=of 1] {1-2} {$\phi_{1,2}$} {} {} ;
  \factor[below right=of 1] {1-3} {$\phi_{1,3}$} {} {} ;
  \factor[below right=of 2] {2-4} {$\phi_{2,4}$} {} {} ;
  \factor[below left=of 3] {3-4} {$\phi_{3,4}$} {} {} ;
  \factoredge[-] {1} {1-2} {2} ; %
  \factoredge[-] {1} {1-3} {3} ; %
  \factoredge[-] {2} {2-4} {4} ; %
  \factoredge[-] {3} {3-4} {4} ; %
  %\edge[-] {2,3} {4} ; %
}

\end{figure}

Denote a particular residential pattern as $\mathbf{y}$.  Every 
residential pattern has a cost that depending on how much it
frustrates individual household racial preferences. 

\begin{align}
\operatorname{C}(\mathbf{y}) = \sum_{<i j>}^{\mathcal{N}}\epsilon_{i,j}(y_i,y_j,\phi_{i,j})
\end{align}

Where $\mathcal{N}$ is the set of pairs of indices of neighboring
blocks and the index of the first block is smaller than the index of
the second block. Also, where

\begin{equation}
  \epsilon_{i,j}(y_i,y_j\phi_{i,j}) = \begin{cases}
    0  &y_i = y_j \\
    \phi_{i,j}  &y_i \neq y_j
  \end{cases}
\end{equation}

Let $\phi_{i,j}=1$ for all $i$ and $j$. Racially segregated
residential patterns have lower costs than integrated patterns
(Table~\ref{table:energy}).

\begin{table}[h]
\begin{tabular}{cl}
  \\
  $\operatorname{E}(\mathbf{y}) = 2$
  & 
  \begin{tabular}{cccc}
    \scalebox{0.5}{
      \tikz{ %
        \node[latent] (1) {$0$} ; %
        \node[latent, below left=of 1] (2) {$0$} ; %
        \node[latent, fill=black, below right=of 1] (3) {\textcolor{white}{$1$}} ; %
        \node[latent, fill=black, below left=of 3] (4) {\textcolor{white}{$1$}} ; %
        \factor[below left=of 1] {1-2} {} {} {} ;
        \factor[below right=of 1] {1-3} {} {} {} ;
        \factor[below right=of 2] {2-4} {} {} {} ;
        \factor[below left=of 3] {3-4} {} {} {} ;
        \factoredge[-] {1} {1-2} {2} ; %
        \factoredge[-] {1} {1-3} {3} ; %
        \factoredge[-] {2} {2-4} {4} ; %
        \factoredge[-] {3} {3-4} {4} ; %
        %\edge[-] {2,3} {4} ; %
      } 
    }
    &
    \scalebox{0.5}{
      \tikz{ %
        \node[latent, fill=black] (1) {\textcolor{white}{$1$}} ; %
        \node[latent, fill=black, below left=of 1] (2) {\textcolor{white}{$1$}} ; %
        \node[latent, below right=of 1] (3) {$0$} ; %
        \node[latent, below left=of 3] (4) {$0$} ; %
        \factor[below left=of 1] {1-2} {} {} {} ;
        \factor[below right=of 1] {1-3} {} {} {} ;
        \factor[below right=of 2] {2-4} {} {} {} ;
        \factor[below left=of 3] {3-4} {} {} {} ;
        \factoredge[-] {1} {1-2} {2} ; %
        \factoredge[-] {1} {1-3} {3} ; %
        \factoredge[-] {2} {2-4} {4} ; %
        \factoredge[-] {3} {3-4} {4} ; %
        %\edge[-] {2,3} {4} ; %
      } 
    }
    &
        \scalebox{0.5}{
      \tikz{ %
        \node[latent] (1) {$0$} ; %
        \node[latent, fill=black, below left=of 1] (2) {\textcolor{white}{$1$}} ; %
        \node[latent, below right=of 1] (3) {$0$} ; %
        \node[latent, fill=black, below left=of 3] (4) {\textcolor{white}{$1$}} ; %
        \factor[below left=of 1] {1-2} {} {} {} ;
        \factor[below right=of 1] {1-3} {} {} {} ;
        \factor[below right=of 2] {2-4} {} {} {} ;
        \factor[below left=of 3] {3-4} {} {} {} ;
        \factoredge[-] {1} {1-2} {2} ; %
        \factoredge[-] {1} {1-3} {3} ; %
        \factoredge[-] {2} {2-4} {4} ; %
        \factoredge[-] {3} {3-4} {4} ; %
        %\edge[-] {2,3} {4} ; %
      }
    }
    &
    \scalebox{0.5}{
      \tikz{ %
        \node[latent, fill=black] (1) {\textcolor{white}{$1$}} ; %
        \node[latent, below left=of 1] (2) {$0$} ; %
        \node[latent, fill=black, below right=of 1] (3) {\textcolor{white}{$1$}} ; %
        \node[latent, below left=of 3] (4) {$0$} ; %
        \factor[below left=of 1] {1-2} {} {} {} ;
        \factor[below right=of 1] {1-3} {} {} {} ;
        \factor[below right=of 2] {2-4} {} {} {} ;
        \factor[below left=of 3] {3-4} {} {} {} ;
        \factoredge[-] {1} {1-2} {2} ; %
        \factoredge[-] {1} {1-3} {3} ; %
        \factoredge[-] {2} {2-4} {4} ; %
        \factoredge[-] {3} {3-4} {4} ; %
        %\edge[-] {2,3} {4} ; %
      }
    }

    \\
  \end{tabular}
  \\ \hline

  $\operatorname{E}(\mathbf{y}) = 4$
  &
  \begin{tabular}{cccc}
    \scalebox{0.5}{
      \tikz{ %
        \node[latent] (1) {$0$} ; %
        \node[latent, fill=black, below left=of 1] (2) {\textcolor{white}{$1$}} ; %
        \node[latent, fill=black, below right=of 1] (3) {\textcolor{white}{$1$}} ; %
        \node[latent, below left=of 3] (4) {$0$} ; %
        \factor[below left=of 1] {1-2} {} {} {} ;
        \factor[below right=of 1] {1-3} {} {} {} ;
        \factor[below right=of 2] {2-4} {} {} {} ;
        \factor[below left=of 3] {3-4} {} {} {} ;
        \factoredge[-] {1} {1-2} {2} ; %
        \factoredge[-] {1} {1-3} {3} ; %
        \factoredge[-] {2} {2-4} {4} ; %
        \factoredge[-] {3} {3-4} {4} ; %
        %\edge[-] {2,3} {4} ; %
      }
    }
    &
    \scalebox{0.5}{
      \tikz{ %
        \node[latent, fill=black] (1) {\textcolor{white}{$1$}} ; %
        \node[latent, below left=of 1] (2) {$0$} ; %
        \node[latent, below right=of 1] (3) {$0$} ; %
        \node[latent, fill=black, below left=of 3] (4) {\textcolor{white}{$1$}} ; %
        \factor[below left=of 1] {1-2} {} {} {} ;
        \factor[below right=of 1] {1-3} {} {} {} ;
        \factor[below right=of 2] {2-4} {} {} {} ;
        \factor[below left=of 3] {3-4} {} {} {} ;
        \factoredge[-] {1} {1-2} {2} ; %
        \factoredge[-] {1} {1-3} {3} ; %
        \factoredge[-] {2} {2-4} {4} ; %
        \factoredge[-] {3} {3-4} {4} ; %
        %\edge[-] {2,3} {4} ; %
      } 
    }
    \end{tabular}
\end{tabular}

\caption{Costs of Neighborhood Patterns}
\label{table:energy}
\end{table}

\subsection*{Finding a Compatible Pattern}
In our little, four-block city, we can look at all six possible
residential patterns and score each one individually to find the
patterns that are most compatible with individual racial
preferences. For larger cities, we will not be able to do this. In a
city with 50 white household and 50 black households, there are
approximately 125 trillion residential patterns (assuming one
household per block and no vacant blocks).

This combinatorial explosion means that for even small cities, we
cannot possibly check every possible residential pattern in
human-scale time to find optimum scoring patterns. The great number of
residential patterns also thwarts statistical machinery like Markov
Chain Monte Carlo.\footnote{In order to converge on the mode of the
  distribution of scores, we have to calculate a normalizing constant,
  which is the sum of the scores of all possible patterns. This is
  computationally too expensive. There have been a number of attempts
  to find an acceptable substitute for the normalizing constant, but
  the empirical results have disappointed.\cite{li_mrf_2009}}

However, researchers, largely in the field of computer vision, have
developed methods to quickly find the lowest scoring patterns for
problems like this. In 1986, Greig, Porteous, and Sehult demonstrated
that finding the best scoring residential pattern for a two races 
was equivalent to solving the graph cutting problem of min
cut.\cite{greig_exact_1989}

 Since the classic 1956 result of Ford and Fulkerson, we have known
 how to solve the min cut problem swiftly.\footnote{Ford and Fulkerson's
   original algorithm is not polynomial, but it was quickly improved
   to solve the problem in polynomial time.\cite{ford_maximal_1956}}
 In the 2000s, these graph cut techniques became popular in the
 computer vision community leading to theoretical
 clarification and methodological extension.\cite{kolmogorov_what_2004}

We now have a set of graph cut techniques that allow us to find, or
approximately find, the lowest scoring pattern for score functions
that have the following form.

\begin{align}
\operatorname{E}(\mathbf{y}) = \sum_{<i j>}^{\mathcal{N}}\epsilon_{i,j}(y_i,y_j)
\end{align}

\section*{From Segregation to Racial Preferences}
If we can quickly find a residential pattern that is most compatible
with given racial preferences, we can also \emph{learn} racial
preferences from observed patterns of residential segregation. The
machinery for this is called structured SVM. 

Say we have the observed residential patterns from $M$ cities and we
want to learn how the levels of household racial preferences
vary. Consider a racial frustration cost like

\begin{align}
  \phi_{i,j} = w_0 + w_1\operatorname{I_1}(x_{i}, x_{j}) +
  w_2\operatorname{I_2}(x_{i}, x_{j}) + ... +
  w_{m-1}\operatorname{S_n}(x_{i},x_{j})  
\end{align} 

Where $\operatorname{I_k}$ is an indicator function that takes a value
of 1 if the blocks $x_{i}$ and $x_j$ are in city $k$, and 0 otherwise.

Given this choice of form, learning a city's racial frustration level
means finding the vector of weights $\mathbf{w}: \{w_0, w_1, ..., w_{m-1}\}$ that give our observed residential patterns a lower
cost than any other possible residential pattern.

However, this learning problem does not have a unique
solution. Remember in our four block city, the segregated residential
pattern had the best score when the racial frustration cost was
1. Those same residential patterns would have had the lowest cost if
the frustration cost had been 2 or 1000 or any postive number.

In order to specify a unique solution, we need to add two
constraints. First, we will go beyond requiring that the preferred
assignment have the lowest score. We will attempt to find $\mathbf{w}$
such that our preferred assignment has a lower score than any other
assignment by the largest possible margin. 

Second, we will also put a constraint on $\mathbf{w}$. A
mathematically convenient choice is to require that $\sqrt{\sum_i^M w_i^2 = 1}$.

Se can state the problem as 
%
\begin{align*}
&\argmax_{\mathbf{w}:||\mathbf{w}||=1} \mathbf{\gamma} \\
&\text{such that} \\
&\operatorname{E}(\mathbf{y}, \mathbf{I}, \mathbf{w})
- \operatorname{E}(\mathbf{y}^*, \mathbf{I}, \mathbf{w}) \geq \gamma\\ 
&\text{for all } \mathbf{y} \text{ where } \mathbf{y} \text{ is in the set of
  possible neighborhood assignments}\\
&\text{and } \mathbf{y} \neq \mathbf{y}^*
\end{align*}
%

Here $\mathbf{y}*$ are the observed residential patterns and $\mathbf{I}$
are the city specific indicator functions.

This is a large margin problem, and problems of this form have and
equivalent canonical representation, which is a quadratic programming
problem.

%
\begin{align*}
&\argmin_{\mathbf{w}} \frac{1}{2}||\mathbf{w}||^2 \\
&\text{such that} \\
&\operatorname{E}(\mathbf{y}, \mathbf{I}, \mathbf{w})
- \operatorname{E}(\mathbf{y}^*, \mathbf{I}, \mathbf{w}) \geq 1 \\ 
&\text{for all } \mathbf{y} \text{ where } \mathbf{y} \text{ is in the set of
  possible neighborhood assignments}\\
&\text{and } \mathbf{y} \neq \mathbf{y}^*
\end{align*}


\subsection*{Learning Sketch}
We usually can't directly solve this quadratic program because there
are so many possible assignments, however we can still find the
optimal weights through the following procedure.

Initialize the weights to some starting value. Create an empty set of
constraints $\mathcal{S}$. Then, find the neighborhood assignment that
has the lowest score given those weights. Call this assignment
$\mathbf{y}_1$ and add it to the constraint set $\mathcal{S}$.

Update the weights by solving the quadratic program: 
%
\begin{align*}
&\argmin_{\mathbf{w}} \frac{1}{2}||\mathbf{w}||^2 \\
&\text{such that} \\
&\operatorname{E}(\mathbf{y}, \mathbf{I}, \mathbf{w})
- \operatorname{E}(\mathbf{y}^*, \mathbf{I}, \mathbf{w}) \geq 1 \\ 
&\text{for all } \mathbf{y} \text{ where } \mathbf{y} \text{ is in } \mathcal{S}\\
&\text{and } \mathbf{y} \neq \mathbf{y}^*
\end{align*}
%

Now find the lowest cost residential pattern given the updated
weights. Call this assignment $\mathbf{y}_2$ and add it to the
constraint set $\mathcal{S}$. Continue until the lowest cost
pattern given the weights is either the observed pattern or
already in the constraint set.\cite{szummer_learning_2008}

