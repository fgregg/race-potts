\section*{Racial Preferences to Residential Segregation}
In this section, we will introduce a formalism for describing the
compatibility between residential patterns across a city and
individual racist preferences. 

Let's imagine a very small city consisting of blocks. Only one house
lives on each block. In this city are two white household and two
black households. The white households prefer to live next to whites
and the black household prefer to live next to blacks. 

We can represent this tiny town as a network where each block is
connected to the blocks that front the same street. In this
representation, blocks that are kitty corner are not directly
connected. We'll index the blocks as $1$, $2$, $3$, and $4$.

\begin{figure}[h]
  \centering
  \begin{tabular}{cc}
\tikz{
\draw[help lines] (0,0) grid (2,2);
\node at (0.5, 0.5) {3} ;
\node at (1.5, 1.5) {2} ;
\node at (0.5, 1.5) {1} ;
\node at (1.5, 0.5) {4} ;
}
\\
\tikz{ %
  \node[latent] (1) {$1$} ; %
  \node[latent, below left=of 1] (2) {$2$} ; %
  \node[latent, below right=of 1] (3) {$3$} ; %
  \node[latent, below left=of 3] (4) {$4$} ; %
  \edge[-] {2,3} {1} ; %
  \edge[-] {2,3} {4} ; %
}
\end{tabular}
\end{figure}

In our city, each block is inhabited by only a white household or
only a black household. We will denote the race of the $i$th
block as $y_i$. Between every pair of neighbors, there is a racial
match score $\phi_{i,j}$ indicating how much happier they would be if their
racial preferences were fulfilled.

\begin{figure}[!h]
\centering

\tikz{ %
  \node[latent] (1) {$y_1$} ; %
  \node[latent, below left=of 1] (2) {$y_2$} ; %
  \node[latent, below right=of 1] (3) {$y_3$} ; %
  \node[latent, below left=of 3] (4) {$y_4$} ; %
  \factor[below left=of 1] {1-2} {$\phi_{1,2}$} {} {} ;
  \factor[below right=of 1] {1-3} {$\phi_{1,3}$} {} {} ;
  \factor[below right=of 2] {2-4} {$\phi_{2,4}$} {} {} ;
  \factor[below left=of 3] {3-4} {$\phi_{3,4}$} {} {} ;
  \factoredge[-] {1} {1-2} {2} ; %
  \factoredge[-] {1} {1-3} {3} ; %
  \factoredge[-] {2} {2-4} {4} ; %
  \factoredge[-] {3} {3-4} {4} ; %
  %\edge[-] {2,3} {4} ; %
}

\end{figure}


Denote a particular residential pattern as $\mathbf{y}$.  We can score
every single residential pattern on how much it matches individual
household preferences.

\begin{align}
\operatorname{E}(\mathbf{y}) = \sum_{<i j>}^{\mathcal{N}}\epsilon_{i,j}(y_i,y_j,\phi_{i,j})
\end{align}

Where $\mathcal{N}$ is the set of pairs of indices of neighboring
blocks and the index of the first block is smaller than the index of
the second block. Also, where


\begin{equation}
\epsilon_{i,j}(y_i,y_j\phi_{i,j}) = \begin{cases}
  \phi_{i,j}   &y_i = y_j \\
  0  &y_i \neq y_j
\end{cases}
\end{equation}

Let $\phi_{i,j}=1$ for all $i$ and $j$. Racially segregated
residential patterns have higher scores than integrated patterns (Table~\ref{table:energy}).

\begin{table}[h]
\begin{tabular}{cl}
  \\
  $\operatorname{E}(\mathbf{y}) = 2$
  & 
  \begin{tabular}{cccc}
    \scalebox{0.5}{
      \tikz{ %
        \node[latent] (1) {$0$} ; %
        \node[latent, below left=of 1] (2) {$0$} ; %
        \node[latent, fill=black, below right=of 1] (3) {\textcolor{white}{$1$}} ; %
        \node[latent, fill=black, below left=of 3] (4) {\textcolor{white}{$1$}} ; %
        \factor[below left=of 1] {1-2} {} {} {} ;
        \factor[below right=of 1] {1-3} {} {} {} ;
        \factor[below right=of 2] {2-4} {} {} {} ;
        \factor[below left=of 3] {3-4} {} {} {} ;
        \factoredge[-] {1} {1-2} {2} ; %
        \factoredge[-] {1} {1-3} {3} ; %
        \factoredge[-] {2} {2-4} {4} ; %
        \factoredge[-] {3} {3-4} {4} ; %
        %\edge[-] {2,3} {4} ; %
      } 
    }
    &
    \scalebox{0.5}{
      \tikz{ %
        \node[latent, fill=black] (1) {\textcolor{white}{$1$}} ; %
        \node[latent, fill=black, below left=of 1] (2) {\textcolor{white}{$1$}} ; %
        \node[latent, below right=of 1] (3) {$0$} ; %
        \node[latent, below left=of 3] (4) {$0$} ; %
        \factor[below left=of 1] {1-2} {} {} {} ;
        \factor[below right=of 1] {1-3} {} {} {} ;
        \factor[below right=of 2] {2-4} {} {} {} ;
        \factor[below left=of 3] {3-4} {} {} {} ;
        \factoredge[-] {1} {1-2} {2} ; %
        \factoredge[-] {1} {1-3} {3} ; %
        \factoredge[-] {2} {2-4} {4} ; %
        \factoredge[-] {3} {3-4} {4} ; %
        %\edge[-] {2,3} {4} ; %
      } 
    }
    &
        \scalebox{0.5}{
      \tikz{ %
        \node[latent] (1) {$0$} ; %
        \node[latent, fill=black, below left=of 1] (2) {\textcolor{white}{$1$}} ; %
        \node[latent, below right=of 1] (3) {$0$} ; %
        \node[latent, fill=black, below left=of 3] (4) {\textcolor{white}{$1$}} ; %
        \factor[below left=of 1] {1-2} {} {} {} ;
        \factor[below right=of 1] {1-3} {} {} {} ;
        \factor[below right=of 2] {2-4} {} {} {} ;
        \factor[below left=of 3] {3-4} {} {} {} ;
        \factoredge[-] {1} {1-2} {2} ; %
        \factoredge[-] {1} {1-3} {3} ; %
        \factoredge[-] {2} {2-4} {4} ; %
        \factoredge[-] {3} {3-4} {4} ; %
        %\edge[-] {2,3} {4} ; %
      }
    }
    &
    \scalebox{0.5}{
      \tikz{ %
        \node[latent, fill=black] (1) {\textcolor{white}{$1$}} ; %
        \node[latent, below left=of 1] (2) {$0$} ; %
        \node[latent, fill=black, below right=of 1] (3) {\textcolor{white}{$1$}} ; %
        \node[latent, below left=of 3] (4) {$0$} ; %
        \factor[below left=of 1] {1-2} {} {} {} ;
        \factor[below right=of 1] {1-3} {} {} {} ;
        \factor[below right=of 2] {2-4} {} {} {} ;
        \factor[below left=of 3] {3-4} {} {} {} ;
        \factoredge[-] {1} {1-2} {2} ; %
        \factoredge[-] {1} {1-3} {3} ; %
        \factoredge[-] {2} {2-4} {4} ; %
        \factoredge[-] {3} {3-4} {4} ; %
        %\edge[-] {2,3} {4} ; %
      }
    }

    \\
  \end{tabular}
  \\ \hline

  $\operatorname{E}(\mathbf{y}) = 0$
  &
  \begin{tabular}{cccc}
    \scalebox{0.5}{
      \tikz{ %
        \node[latent] (1) {$0$} ; %
        \node[latent, fill=black, below left=of 1] (2) {\textcolor{white}{$1$}} ; %
        \node[latent, fill=black, below right=of 1] (3) {\textcolor{white}{$1$}} ; %
        \node[latent, below left=of 3] (4) {$0$} ; %
        \factor[below left=of 1] {1-2} {} {} {} ;
        \factor[below right=of 1] {1-3} {} {} {} ;
        \factor[below right=of 2] {2-4} {} {} {} ;
        \factor[below left=of 3] {3-4} {} {} {} ;
        \factoredge[-] {1} {1-2} {2} ; %
        \factoredge[-] {1} {1-3} {3} ; %
        \factoredge[-] {2} {2-4} {4} ; %
        \factoredge[-] {3} {3-4} {4} ; %
        %\edge[-] {2,3} {4} ; %
      }
    }
    &
    \scalebox{0.5}{
      \tikz{ %
        \node[latent, fill=black] (1) {\textcolor{white}{$1$}} ; %
        \node[latent, below left=of 1] (2) {$0$} ; %
        \node[latent, below right=of 1] (3) {$0$} ; %
        \node[latent, fill=black, below left=of 3] (4) {\textcolor{white}{$1$}} ; %
        \factor[below left=of 1] {1-2} {} {} {} ;
        \factor[below right=of 1] {1-3} {} {} {} ;
        \factor[below right=of 2] {2-4} {} {} {} ;
        \factor[below left=of 3] {3-4} {} {} {} ;
        \factoredge[-] {1} {1-2} {2} ; %
        \factoredge[-] {1} {1-3} {3} ; %
        \factoredge[-] {2} {2-4} {4} ; %
        \factoredge[-] {3} {3-4} {4} ; %
        %\edge[-] {2,3} {4} ; %
      } 
    }
    \end{tabular}
\end{tabular}

\caption{Scores of Neighborhood Assignments}
\label{table:energy}
\end{table}

In our little city with two neighborhoods, there are two ways of
assigning blocks to neighborhoods that minimizes the score . In general, for a given set of similarity
measures between blocks, there will be a set of assignments that have
a lowest score. For small networks, we can find these best assignments
by checking all possible patterns of assignments. Unfortunately, this
becomes quickly impractical for larger cities. With two possible
neighborhoods, the number of possible assignments is $2^N$ where N is
the number of blocks.

This permutational explosion means that for even small cities, we
cannot possibly check every possible neighborhood assignment in
human-scale time. For the same reason, traditional Markov Chain Monte
Carlo methods for finding the lowest scoring assignment also prove
intractable.\footnote{In order to converge on the mode of the
  distribution of scores, we have to calculate a normalizing constant,
  which is the sum of the scores of all possible assignments. This is
  computationally too expensive. There have been a number of attempts
  to find an acceptable substitute for the normalizing constant, but
  the empirical results have disappointed.\cite{li_mrf_2009}}

However, researchers, largely in the field of computer vision, have
developed methods to quickly find the lowest scoring assignment for
problems like ours. In 1986, Greig, Porteous, and Sehult demonstrated
that finding the lowest scoring assignment for a two neighborhood case
was equivalent to solving the graph cutting problem of min
cut.\cite{greig_exact_1989}

 Since the classic 1956 result of Ford and Fulkerson, we have known
 how to solve the min cut problem swiftly.\footnote{Ford and Fulkerson's
   original algorithm is not polynomial, but it was quickly improved
   to solve the problem in polynomial time.\cite{ford_maximal_1956}}
 In the 2000s, these graph cut techniques became popular in the
 computer vision community leading to theoretical
 clarification and methodological extension.\cite{kolmogorov_what_2004}

We now have a set of graph cut techniques that allow us to find, or
approximately find, the lowest scoring assignment for score functions
that have the following form.\footnote{We can also use these
  techniques on scoring functions that have a term for the relation
  between the properties of a particular block and the region
  assignment, $\epsilon_i(y_i)$. This is particularly useful when we
  already know something about what meaningful categories should exist.

  \begin{align}
    \operatorname{E}(\mathbf{y}) = \sum_i^N\epsilon_i(y_i) + \sum_{<i j>}^{\mathcal{N}}\epsilon_{i,j}(y_i,y_j)
  \end{align}

  For example, if we had a strong, ecological theory of urban space,
  we might want to identify regions like `the red light district,'
  `the zone of industry`, `the central business district.' We would
  want to pay a large penalty if we assigned a block that had lot of
  immigrants to `the central business district` instead of `the
  immigrant neighborhood.'
}

\begin{align}
\operatorname{E}(\mathbf{y}) = \sum_{<i j>}^{\mathcal{N}}\epsilon_{i,j}(y_i,y_j)
\end{align}

We will extensively use one extension to graph cuts called the
Quadratic Pseudo Boolean Optimization algorithm, or QPBO. This method
allows us to approximately find the lowest scoring assignment even
though our scoring function is not submodular. 

A submodular scoring function would be one where $\epsilon_{i,j}(0,0)
+ \epsilon_{i,j}(1,1) \leq \epsilon_{i,j}(0,1) + \epsilon_{i,j}(1,0)$
for all $i,j$ in $\mathcal{N}$. If we had a submodular scoring
function, we would have a theoretical guarantee of finding the lowest
scoring assignment. Our scoring function is decisively not submodular
because we want very dissimilar neighboring blocks to be assigned to
different neighborhoods. That is, for very dissimilar blocks we want
it to hold that $\epsilon_{i,j}(0,0) + \epsilon_{i,j}(1,1)
\boldsymbol{>} \epsilon_{i,j}(0,1) + \epsilon_{i,j}(1,0)$.


\section*{Learning similarities}
Let's consider similarity measures that take the following linear form

\begin{align}
\phi_{i,j} = w_0 + w_1\operatorname{S_1}(x_{i,1}, x_{j,1}) +
w_2\operatorname{S_2}(x_{i,2}, x_{j,2}) + ... +
w_n\operatorname{S_n}(x_{i,m},x_{j,m})
\end{align} 

Where $x_{i,k}$ is the $k$th feature of block $i$. These features can
be scalars or vectors of numbers and could correspond to attributes
like population or distribution of race and ethnicity. A function
$\operatorname{S}$ is similarity or distance function that maps two
features to a real number. These functions must be symmetric; for each
$\operatorname{S}$ and features $a$ and $b$, $\operatorname{S}(a,b) =
\operatorname{S}(b,a)$.

Given this choice of form, learning the similarities between blocks
means finding the vector of weights $\mathbf{w}: \{w_0, w_1, ...,
w_{m-1}, w_m\}$ that produce the aggregate similarities $\phi$'s which
give our target neighborhood assignment the best score. 

However, this learning problem does not have a unique solution. In our
four block city, our preferred neighborhood assignment had the best
score with block-by-block similarities of 1, -1, -1, 1. That
assignment would also have had the best score if the similarities had
been 2, -2, -2, 2 or 1000, -1000, -1000,
1000. 

In order to specify a unique solution, we need to add two
constraints. First, we will go beyond requiring that the preferred
assignment have the lowest score. We will attempt to find $\mathbf{w}$
such that our preferred assignment has a lower score than any other
assignment by the largest possible margin. 

Second, we will also put a constraint on $\mathbf{w}$. A
mathematically convenient choice is to require that $\sqrt{\sum_i^M w_i^2 = 1}$.

So, now we can state the problem as 
%
\begin{align*}
&\argmax_{\mathbf{w}:||\mathbf{w}||=1} \mathbf{\gamma} \\
&\text{such that} \\
&\operatorname{E}(\mathbf{y}, \mathbf{s}, \mathbf{w})
- \operatorname{E}(\mathbf{y}^*, \mathbf{s}, \mathbf{w}) \geq \gamma\\ 
&\text{for all } \mathbf{y} \text{ where } \mathbf{y} \text{ is in the set of
  possible neighborhood assignments}\\
&\text{and } \mathbf{y} \neq \mathbf{y}^*
\end{align*}
%

Here $\mathbf{y}*$ is the target neighborhood assignment and $\mathbf{s}$
are similarity measures between blocks. 

This is a large margin problem, and problems of this form have and
equivalent canonical representation, which is a quadratic programming
problem.

%
\begin{align*}
&\argmin_{\mathbf{w}} \frac{1}{2}||\mathbf{w}||^2 \\
&\text{such that} \\
&\operatorname{E}(\mathbf{y}, \mathbf{s}, \mathbf{w})
- \operatorname{E}(\mathbf{y}^*, \mathbf{s}, \mathbf{w}) \geq 1 \\ 
&\text{for all } \mathbf{y} \text{ where } \mathbf{y} \text{ is in the set of
  possible neighborhood assignments}\\
&\text{and } \mathbf{y} \neq \mathbf{y}^*
\end{align*}


\subsection*{Learning Sketch}
We usually can't directly solve this quadratic program because there
are so many possible assignments, however we can still find the
optimal weights through the following procedure.

Initialize the weights to some starting value. Create an empty set of
constraints $\mathcal{S}$. Then, find the neighborhood assignment that
has the lowest score given those weights. Call this assignment
$\mathbf{y}_1$ and add it to the constraint set $\mathcal{S}$.

Update the weights by solving the quadratic program: 
%
\begin{align*}
&\argmin_{\mathbf{w}} \frac{1}{2}||\mathbf{w}||^2 \\
&\text{such that} \\
&\operatorname{E}(\mathbf{y}, \mathbf{s}, \mathbf{w})
- \operatorname{E}(\mathbf{y}^*, \mathbf{s}, \mathbf{w}) \geq 1 \\ 
&\text{for all } \mathbf{y} \text{ where } \mathbf{y} \text{ is in } \mathcal{S}\\
&\text{and } \mathbf{y} \neq \mathbf{y}^*
\end{align*}
%

Now find the lowest energy neighborhood assignment given the updated
weights. Call this assignment $\mathbf{y}_2$ and add it to the
constraint set $\mathcal{S}$. Continue until the lowest energy
assignment given the weights is either the target assignment or
already in the constraint set.\cite{szummer_learning_2008}


\subsection*{Empirical Loss}
Some readers may have noticed that the our four-city block example
still does not have a unique solution for the learning problem we have
set up. There are two different neighborhood assignments that have the
same lowest score, and they both achieve our goal of putting similar
blocks in the same neighborhood and dissimilar blocks in different
neighborhoods (Table \ref{table:lowest}).

\begin{table}
\centering
  \begin{tabular}{cc}
      \tikz{ %
        \node[latent] (1) {$0$} ; %
        \node[latent, below left=of 1] (2) {$0$} ; %
        \node[latent, fill=black, below right=of 1] (3) {\textcolor{white}{$1$}} ; %
        \node[latent, fill=black, below left=of 3] (4) {\textcolor{white}{$1$}} ; %
        \factor[below left=of 1] {1-2} {} {} {} ;
        \factor[below right=of 1] {1-3} {} {} {} ;
        \factor[below right=of 2] {2-4} {} {} {} ;
        \factor[below left=of 3] {3-4} {} {} {} ;
        \factoredge[-] {1} {1-2} {2} ; %
        \factoredge[-] {1} {1-3} {3} ; %
        \factoredge[-] {2} {2-4} {4} ; %
        \factoredge[-] {3} {3-4} {4} ; %
      } 
    &
      \tikz{ %
        \node[latent, fill=black] (1) {\textcolor{white}{$1$}} ; %
        \node[latent, fill=black, below left=of 1] (2) {\textcolor{white}{$1$}} ; %
        \node[latent, below right=of 1] (3) {$0$} ; %
        \node[latent, below left=of 3] (4) {$0$} ; %
        \factor[below left=of 1] {1-2} {} {} {} ;
        \factor[below right=of 1] {1-3} {} {} {} ;
        \factor[below right=of 2] {2-4} {} {} {} ;
        \factor[below left=of 3] {3-4} {} {} {} ;
        \factoredge[-] {1} {1-2} {2} ; %
        \factoredge[-] {1} {1-3} {3} ; %
        \factoredge[-] {2} {2-4} {4} ; %
        \factoredge[-] {3} {3-4} {4} ; %
      } 
    \\
  \end{tabular}
  \caption{Equivalent Lowest Assignments}
  \label{table:lowest}
\end{table}

Happily, and for unrelated reasons, practicioners have found that good
performance for this type of learning problem depends upon incorporating an
empirical loss function into the objective function. If we follow
suit, we will also side step this problem of multiple equivalent
assignments. 

Below we show how to incorporate the empirical loss:
%
\begin{align*}
&\argmin_{\mathbf{w}} \frac{1}{2}||\mathbf{w}||^2 +
  C\cdot\xi\\
&\text{such that} \\
&\operatorname{E}(\mathbf{y}, \mathbf{s}, \mathbf{w})
- \operatorname{E}(\mathbf{y}^*, \mathbf{s}, \mathbf{w}) \geq
\Delta(\mathbf{y}^*, \mathbf{y}) - \xi\\ 
& \xi \geq 0\\
&\text{for all } \mathbf{y} \text{ where } \mathbf{y} \text{ is in the set of
  possible neighborhood assignments,}\\
&\xi \text{ is a slack variable, and } \Delta \text{ is an empirical loss.}\\
\end{align*}
%

The key to solving this problem is finding assignments that have the
lowest score, where the score incorporates the empirical loss: i.e.
%
\begin{align}
\operatorname{E}(\mathbf{y}) = \sum_{<i j>}^{\mathcal{N}}\epsilon_{i,j}(y_i,y_j) + C\cdot\Delta(\mathbf{y}*,
\mathbf{y})
\end{align}
%
If the loss can be decomposed over the blocks like
%
\begin{align}
\operatorname{E}(\mathbf{y}) = \sum_{<i j>}^{\mathcal{N}}\epsilon_{i,j}(y_i,y_j) + C\cdot\sum_i^N\Delta(y_i*, y_i)
\end{align}
%
then the objective has the form that allows us to use an algorithm
like QPBO to quickly find an approximate minimum assignment. If the
loss function does not decompose we have to check an exponentially
many possible assignment, typically an intractable problem.

We want a loss function that increases the further we are from our
target assignment. A natural choice is to penalize every block that is
incorrectly assigned. If we penalize every error the same then we have
the classic Hamming loss. 

$\sum_i^N\begin{cases}
  0 &y_i^* = y_i \\
  1 &y_i^* \neq y_i
\end{cases}$. 

Incorporating this empirical loss into our learning problem means
that, of a set of previously equivalent assignments, only the
assignment closest to the training data will be have the lowest score.
